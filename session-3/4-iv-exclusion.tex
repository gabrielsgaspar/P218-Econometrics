% Begin section ----------------------------------------------------------------
\section{Exclusion Restriction}

% Part (a) ---------------------------------------------------------------------
\begin{frame}{Exclusion Restriction}

    We saw that we need relevance and exclusion in order to compute our IV estimator. The exclusion restriction says that
    \begin{align*}
        \mathbb{E}[\varepsilon_i Z_i \mid W_i] = 0
    \end{align*}
    Can we test this? What if we use $\hat{\varepsilon}_i$ as a replacement for $\varepsilon_i$?
    
\end{frame}

\begin{frame}{Exclusion Restriction}

    The exclusion restriction is something that is untestable! So if you are using IV for one of your papers, you should motivate why this r estriction should hold.
    
    \vspace{2em}
    
    Let's think of a case where your instrument is virtually (near) random (e.g. a lottery, weather, etc.) and the relevance restriction is satisfied. Does that mean that we can take a deep breath and assume that the exclusion restriction holds?
\end{frame}

\begin{frame}{Three examples}

    Let's discuss three examples:
    \begin{enumerate}
        \item The Vietnam War lottery: let $y_i$ be death, $viet_i$ is whether a person served in the Vietnam war and our instrument $z_i$ is their lottery number.
        \item Rain as an instrument: let $y_i$ be conflict, $incom_i$ is income and $z_i$ is the amount of rain in location $i$.
        \item Legal origin: let $y_i$ be some development indicator, $institut_i$ is the presence of some institution and $z_i$ is the legal origin of a given country.
    \end{enumerate}
\end{frame}

\begin{frame}{A Framework}

    How can we think about the exclusion restriction then when reading a paper or doing our own research? Check out Angrist, Imbens and Rubin (1996) for proof:
    \begin{align*}
        \frac{\mathbb{E}[Y_i (1, D_i(1) - Y_i(0, D_i(0)]}{\mathbb{E}[D_i(1) - D_i(0)]} = &\mathbb{E}[Y_i (1, D_i(1) - Y_i(0, D_i(0) \mid i \text{ complier}]
        \\
        &+ \mathbb{E}[H_i \mid i \text{ non-complier}] \times \frac{\operatorname{Pr}(i \text{ non-complier})}{\operatorname{Pr}(i \text{ complier})}
    \end{align*}
    and $H_i = Y_i(1, d) - Y_i(0, d)$ where $d = 1$ for always-taker and $d=0$ for never-taker.
\end{frame}